%%%%%%%%%%%%%%%%%%%%%%%%%%%%%%%%%%%%%%%%%%%%%%
%% Created by John Paul Minda, PhD			%%
%% Professor of Psychology					%%
%% The Brain and Mind Institute				%%
%% The University of Western Ontario		%%	
%% London, ON N6A 5C2						%%
%%											%%
%% Version 1.2								%%	
%% Feb 13, 2018								%%
%%%%%%%%%%%%%%%%%%%%%%%%%%%%%%%%%%%%%%%%%%%%%%

\documentclass{article}
\usepackage{fullpage}

\renewcommand{\familydefault}{\sfdefault}
\usepackage[scaled=1]{helvet}
\usepackage[helvet]{sfmath}
\everymath={\sf}

\usepackage{parskip}
\usepackage[colorinlistoftodos]{todonotes}
\usepackage[colorlinks=true, allcolors=blue]{hyperref}

\title{Categorization Lab Manual}
\author{John Paul Minda, PhD}
%\company{The University of Western Ontario}
\setcounter{tocdepth}{2}
\begin{document}
\maketitle
\tableofcontents

\section{Introduction} 
I have created this document as a way to summarize my expectations for students and trainees, to outline my role as a supervisor, to describe my research philosophy, and to detail aspects of how the lab works. This is a combination of a practical guide and a treatise on how I think a lab should be run. It is not my intention to be a micro manager. I do not like to micro manage and I don't do it well. Everything in this document is intended to be a guideline and should not be interpreted as a strict policy. Except for the part where I said that it should not be interpreted as a strict policy. That part you should take seriously.

The second section is probably the most important, because I describe our roles and expectations. We're all here to study Cognitive Science: How people think, how they learn, how they behave, and how the brain operates to support these functions. In order to achieve this goal, it is important to consider our roles in the lab. Doing so will also help in completing successfully your degree or program. 

I endeavour to treat all my students, trainees, and volunteers with respect and courtesy. I value diversity and equity in my lab. If you are are aware of or are experiencing harassment of any kind, please either bring this to my attention and/or discuss with the university \href{http://www.uwo.ca/equity/}{Equity Office}. 

\subsection{About this Lab}
Why are we named \textbf{The Categorization Lab}? I chose that name for the lab in 2003 because I'm interested in the process of categorization. I'm interested in other things, if course, but that's at the core. How and why to humans (and non human animals) categorize and classify things? We are part of the Department of Psychology at Western and also affiliated with the Brain and Mind Institute at Western. All lab members except collaborators will be members of the Brain and Mind institute as well. 



\section{Expectations and Roles} 
I supervise undergraduate volunteers, undergraduate honours thesis projects, Master's students, Doctoral students, and post docs. My role as supervisor varies according to the specific role of the trainee. Below are some specifics with respect to my role and also the expectations and roles for each category of trainee. Many of these things overlap (e.g. attendance at lab meetings is expected for many trainees) but in other cases the expectations are specific (e.g. senior PhD students are often expected to play a supervisory role with respect to honours students). 

\subsection{Supervisor Roles}
My role is to supervise, to guide, and to advise. I have my own research interests, but my primary role to be the lab's director. I am not a boss and do not expect to be an independent researcher above and beyond the scope of the lab. Rather, I set the research agenda and strive to create a culture where all of our ideas and suggestions are valued. I expect that we are all here to engage in fundamental scientific research and are motivated to discover how people think, behave, how they learn about categories, and how conceptual structure influences thinking and behaviour. 

%should I say something about the need to stay within the confines of categorization?

\subsubsection{Research Guidance} As a supervisor, I will help junior students and undergraduate students decide on, design, and implement their research projects. I expect to guide senior students in scientific discovery and professional development. In this role, I will read your work. I will edit your documents. I will offer critiques and suggestions and may suggest analyses. My suggestions are intended to be helpful as well as critical and you should consider them, but you are not under the obligation to necessarily implement them.  I will challenge you from time to time, and may disagree with how you interpret your work or design a study. And I can be challenged, argued with, and convinced by evidence that I'm wrong (which is not infrequent).

\subsubsection{Academic Advising} If you are a graduate student in my lab, I will serve as your academic supervisor and advisor. That means that I will help you decide on course selection, serve as a \textit{de facto} member of your advisory committee, sign off on official documents, and will work with you to ensure that you are aware of and meet your program requirements. I can also help as a resource when making career decisions. 

\subsubsection {Letters of Reference} For all trainees: I will act as a professional reference for you. This could be in the form of letters of reference for employment, for training opportunities, graduate school, grants, and awards. If you are seeking a letter of reference, please notify me as soon as possible and as far in advance as possible (several weeks or months). Please provide me with as much information as possible, including your CV, reference forms, reminders, etc. It can help to provide me a list of things you would like me to mention in the letter. The more information I have, the better letter I can write for you. Please bear in mind that I may have several (many) other letters to write, including those for students outside the lab. As a rule, I will write your letter personally and ensure that it arrives when and where it needs to.

\subsubsection{Funding} Although my ability to fund students depends on my own success with external grants, I am committed to funding you as I am able. As a graduate student, you will probably receive a tuition scholarship as well as a GTA for fall and winter. You may also receive funding in the form of an external award (NSERC, SSHRC, CIHR, or OGS). The details of departmental funding can be found on the \href{http://psychology.uwo.ca/graduate/index.html}{department website}. My grant-funded support may cover some of the following:

\begin{itemize}
\item I can provide summer funding for graduate students at a rate that is similar to the GTA. If you are receiving a scholarship (Federal, Provincial, or University) I will not provide additional summer funding or salary beyond that.
\item I can subsidize at least 1 conference trip per year for graduate students.
\item I will cover the costs associated with research projects for all trainees. 
\item I provide all the computing and equipment resources necessary for your research.
\item I will cover the costs associated with publication, copying, and printing posters.
\end{itemize}

%This could be moved to a separate section

\subsection {Research Assistants and Volunteers}
Volunteer and work study research assistants work in the lab doing primary data collection, scheduling research participants, doing literature searches, scoring questionnaires, and engaging in other similar tasks. Research volunteers usually work under the supervision of graduate students. Time commitment is typically 5-10 hours per week, and I encourage you to see if you qualify for work study. If you qualify, I will create a position for you and hire you to that position. See Western's \href{https://workstudy.uwo.ca/}{Work Study Office} for more detail. 

As a research assistant and/or volunteer, you will have key card access to the office space areas in WIRB building and can request a key for access to the SSC testing rooms.
 
\subsection {Honours Students}
The lab supervises 1-3 undergraduate honours students each year. As an honours student, you are expected to carry out an original research project that is related to one of the main themes in the lab. Typically, I will have several projects in development that you can work on. Honours students are often supervised by senior graduate students or post docs in the lab. I will provide final oversight, subsidize the cost of printing your poster. If you are submitting the resulting research project to a journal for publication, a graduate student and/or I will act as senior, corresponding author (final author) on any journal submissions or conference proceedings.

\subsubsection{Milestones and Duties}
In addition to the general requirements of the \href{http://psychology.uwo.ca/undergraduate/honors_thesis/index.html} {honours thesis course},there are several general milestones and duties that I ask honours students to carry out. We will discuss all of these things at our initial meetings 


\begin{description}
\item [Lab Meeting Attendance] - I expect honours thesis students to attend the weekly lab meetings if possible, though I recognize that these meetings may conflict with your classes. See the section on \nameref{sec:Lab} for more information on lab meetings.

\item [Project Selection] - Students should select a specific project in the first two weeks of the term. Most projects will be part of the overall focus of the lab, and may be related to a project that is being conducted by a graduate student. In this case, you will work with the graduate student on the project. 

\item [Annotated Bibliography] - Students should complete an annotated bibliography for their project that includes 10-15 papers that make up the core literature on the topic. The format is variable, but it is typically a list of each reference followed by a 1-2 paragraph summary of the research and why it is important to the current study. 

\item [Practice Talk] - Students should give a practice version of their thesis talks in one of our lab meetings if possible

\item [Practice Poster] - Students should give a practice version of their poster in one of our lab meetings if possible
\end{description}

\subsection {MSc Students}
Master's students are bound by the department's \href{http://psychology.uwo.ca/graduate/index.html}{general guidelines}. Most students in my lab are enrolled in the Cognitive Developmental and Brain Sciences (CDBS) program and should follow the specific guidelines for the \href{http://psychology.uwo.ca/graduate/program_information/cdbs_program_requirements.html}{CCDBS graduate program}. My general expectations for MSc students is that you will be active participants in the lab. That means that you are spending time in the lab, you are attending lab meetings, CDBS area seminars, departmental talks. 

Most MSc students are planning to apply to the PhD program, and that is my expectation as well. This is not a requirement, though, and acceptance into the PhD program is not guaranteed. There are many good reasons to stay and pursue a PhD and there are also many good reasons not to pursue a PhD. We will discuss these things during the course of your study and there is a section below on making this decision.

\subsubsection{Milestones and Duties}

\begin{description}
\item [Lab Meeting Attendance] - I expect Master's students to attend lab meetings as often as possible and to present research several times each term. See the section on \nameref{sec:Lab} for more information.

\item [Individual Meetings] - I meet with Master's students individually at least once every other week. These meetings can be 30 min to an hour and we will discuss your projects, program, course work, and plans. We'll decide on a meeting time that works for us, and we should both protect the time from encroachment by other duties. 

\item [Initial Project] - In the first year in the lab, you will be assigned an initial project. This project will be related to one of our primary research themes and we will work together to develop the idea into an empirical research project. Your responsibility is to read and master the relevant background literature, to help design the empirical protocol, to apply for or amend ethics protocols (see the section on \nameref{sec:Ethics}), to collect the data and to run the primary analyses. I will assist with any of these aspects, as will senior students. 

\item [Seminar Talks] - It is your responsibility to present your research (proposed, in progress, and/or completed) at the weekly seminar meetings held by the \href{http://psychology.uwo.ca/graduate/program_information/cdbs_program_requirements.html} {CDBS} area. These talks are always held Friday afternoon at 12:30 during the fall and winter terms. Your talk can be based on research in progress, research completed, or research that you are proposing. Please plan to give a practice talk in a lab meeting prior to presenting in the CDBS seminar. 

\item [Conference Attendance] - Conference attendance is encouraged for all graduate students. This is a chance to present your work to the scientific community and to network with other cognitive scientists. See the section on \nameref{sec:Conf} for more information.

\item [Master's Thesis] - The capstone to your master's program is the master's thesis. The master's thesis should be an experimental and/or computational project that is designed by you, with my input. In our lab, the most common format is an experimental paper on some aspect of categorization, higher-order cognition, or mindfulness meditation. Other topics are possible, but the central work should still be within the range of topics that are being investigated in our lab. An experimental thesis will typically contain a literature review introduction, a full write up of 2 or more experiments that you designed, conducted, and analyzed, and your interpretation of those results. The work you undertake in the Initial Project (described above) may or may not evolve into a Master's Thesis. Though it is not a requirement of the Master's program, you should consider preregistering your master's thesis work with Open Science Foundation (see the section on \nameref{sec:LabPrac} for more information on OSF.

\item [Other Research] - Although it is not a formal requirement of the MSc program, students can and should be involved in other research as well. This can take many forms. You can assist on my research by learning to program behavioural studies in PsychoPy, by scheduling research participants, running experiments, conducting basic analyses, etc. You can do the same on projects being led by senior PhD students or projects with honours students. \textit{The best way to improve on your research ability and skill is to keep doing it.} The best way to understand more about the brain and mind is to work on research projects and to think about research projects that test the predominate theories in our field. 

\item [Publication of Thesis] - You should publish and/or present your thesis if possible. The outlet will depend on the topic and also om the outcome of the experiments. Preregistration will facilitate the process by having some initial peer review of the project. You should discuss this with me as you are designing the study.

\item [Decisions About Doctoral Work] - During the summer between your first and second year, we will discuss doctoral work. The PhD program at Western requires a Master's degree and the transition from Master's to PhD is a natural progression. But the PhD is not for everyone. You should be prepared to work on Cognitive Psychology, Cognitive Science, and Cognitive Neuroscience for the next 4-5 years (and well beyond). You should think about the these questions as you contemplate your decision: 
\begin{itemize}
\item Do you want to commit the next half of a decade to being in my lab? 
\item Do you want to work on the same projects for month or years? 
\item Are you passionate about Cognitive Psychology, Cognitive Science, and Cognitive Neuroscience? 
\end{itemize}

You want to be able to answer ``Yes'' to these questions. If not, you may want to consider stopping with a Master's Degree. These are all things we should discuss, and you should know that I do not wish to pressure anyone into a PhD program. I can assist and advise as you make the best decision for yourself. I want to say clearly that if you are considering the option of completing your MSc and not pursuing a PhD, I will be supportive. Please do not feel like you are letting me down or letting the program down.  
\end{description}


\subsection {PhD Students}
Doctoral students are bound by the department's \href{http://psychology.uwo.ca/graduate/index.html}{general guidelines}. Most students in my lab are enrolled in the Cognitive Developmental and Brain Sciences (CDBS) program and should follow the specific guidelines for the \href{http://psychology.uwo.ca/graduate/program_information/cdbs_program_requirements.html}{CCDBS graduate program}. Other students in my lab may be enrolled in the Clinical Psychology program or the Neuroscience program. I expect PhD students to be active participants in my lab but to be working toward independence as well. That means that you are spending time in the lab, your are attending and leading lab meetings, attending CDBS area seminars, attending departmental talks. As a PhD student, you should also take the opportunity to help mentor and supervise undergraduate students in the lab as they work on their honours theses, to supervise undergraduate research volunteer and to help to mentor more junior graduate students.

\subsubsection{Milestones and Duties}
Many of these items are similar to to the master's category, but the level of commitment and expectations are higher. 

\begin{description}

\item [Lab Meeting Attendance] - I expect PhD students to attend lab meetings as often as possible. See the section on \nameref{sec:Lab} for more information.

\item [Individual Meetings] - I usually meet with PhD students individually every other week. These meetings can be 30 min to an hour (or more) and we will discuss your projects, program, course work, and plans. We'll decide on a meeting time that works for us, and we should both protect the time from encroachment by other duties. Consider these meetings to be a regular occurrence throughout the year. Of course we may meet more often if we're working on a specific project with a deadline. On those occasions when we can't meet in person, we can arrange for a virtual meeting over Slack or Skype.

\item [Seminar Talks] - It is your responsibility to present your research (proposed, in progress, and/or completed) at the weekly seminar meetings held by the \href{http://psychology.uwo.ca/graduate/program_information/cdbs_program_requirements.html} {CDBS} area. These talks are always held Friday afternoon at 12:30 during the fall and winter terms. Your talk can be based on research in progress, research completed, or research that you are proposing. Please plan to give a practice talk in a lab meeting prior to presenting in the CDBS seminar. 

\item [Choosing and Advisory Committee] - During your first term, you should choose advisory committee. The advisory committee consists of me (the supervisor) and at at least two other faculty members, at least one of whom is a member of the psychology department. Details are found in the \href{http://psychology.uwo.ca/graduate/program_information/cdbs_program_requirements.html} {CDBS} program requirements.

\item[Dissertation Proposal] - Students in the CDBS program should plan to present a dissertation proposal in their first year. (see the CDBS requirements). The reason for doing the proposal early is to allow the advisory committee to be well informed and involved in the the scope of the project.

\item [Comprehensive Exams] - Comprehensive or qualifying exams are written in the second year of the PhD program. You must pass your qualifying exams to continue in the program. As with many of the programmatic aspects of the PhD, the details are found in the in the \href{http://psychology.uwo.ca/graduate/program_information/cdbs_program_requirements.html} {CDBS} documents. 

\item [Conference Attendance] - I encourage conference attendance is encouraged for all graduate students. This is a chance to present your work to the scientific community and to network with other cognitive scientists. See the section on \nameref{sec:Conf} for more information. You should try to present your work at least once a year.

\item [Dissertation] - The capstone to your PhD, of course, is the dissertation. This is to be an original research project that should be an experimental and/or computation project that is primarily designed and carried out by you, with input and supervision from me. In our lab, the most common format is an experimental paper on some aspect of categorization, higher-order cognition, or mindfulness meditation. Other topics are possible, but the central work should still be within the range of topics that are being investigated in our lab. An experimental thesis will typically contain a literature review introduction, a full write up of several experiments that you designed, conducted, and analyzed, and your interpretation of those results. The format is described by the  

Though it is not a requirement of the PhD program, you should consider preregistering your master's thesis work with Open Science Foundation (see the section on \nameref{sec:LabPrac} for more information on OSF.

\item [Other Research] - Although it is not a formal requirement of the PhD program, students can and should be involved in other research as well. This can take many forms. You can assist on my research by learning to program behavioural studies in PsychoPy, by scheduling research participants, running experiments, conducting basic analyses, etc. You can do the same on projects being led by senior PhD students or projects with honours students. The best way to improve on your research ability and skill is to keep doing it. The best way to understand more about the brain and mind is to work on research projects and to think about research projects that test the predominate theories in our field.

\item [Publication of Thesis] - You should publish and/or present your thesis if possible. The outlet will depend on the topic and also om the outcome of the experiments. Preregistration will facilitate the process by having some initial peer review of the project. 

\item [Supervision of Undergraduates] - PhD students are encouraged to supervise honours students as well as undergraduate RAs. I can provide insight and guidance if needed.

\item [Leadership Opportunities] - There are many opportunities to be involved in leadership, such as helping to coordinate the BMI coffee breaks, helping to coordinate the speaker series, conference organization, and student groups. Ask your peers, other faculty, and me about these if you are not sure. 

\item [Planning for the Future] - The professional landscape for PhDs in psychology or neuroscience is complex and can be daunting. We should be discussing your plans at our meetings and you should also be discussing these things with your peer group both within the university and the program, but also with the cohort of other students in other programs.

One thing to consider is that the majority of PhD students (most of your grad school peers here and elsewhere) do not end up in tenure track faculty positions, but rather working in other industries. There are opportunities with hi-tech companies, consulting firms, government groups, and university research. If you are considering an academic career, your focus needs to be on developing a line of research that you are passionate about, to publish and present in that area, to attract funding in that area, and to seek out additional training as a Postdoc. If you are considering a career in scientific research outside the academic world, the planning is similar, but you should also be looking at developing nontransferable skills (data science skills, programming, analysis, etc.) that you can highlight on your CV. You should also look at post docs fellowships that offer internship opportunities like the \href{https://www.mitacs.ca/en}{Mitacs} program. In either case \textit{talk to other graduate students and post docs about their experiences.} Do not be afraid to ask for help and guidance from me, from your advisory committee, from the grad provost's office (SGPS), and from other organizations.  

One thing that I do want to be clear about is that I will supportive of your decisions. This is your career. You should not feel that your decision to choose a certain career path will run counter to my role as your supervisor. And if you feel that I'm not providing the answers or guidance that you seek, we'll look for other solutions. 


\end{description}

\subsection {Post Docs}
I have very little in the way of explicit expectations for post docs so far, but this section will be expanded. So far, I have only had one postdoc in my lab via MITACS funding. 

%LAB AND OFFICE%

\section {Lab and Office Space}
I provide room for all research activities (data collection, analysis, writing, etc) as well as room for a personal work space. The lab is located in the \href{http://www.uwo.ca/bmi/about/wirb.html}{Western Interdisciplinary Research Building}. Offices are located on the fifth floor. My office is in room 5158. Graduate students have a cubicle in the open office area (5115), post docs will have a semi-enclosed, shared office, and there is temporary space for undergraduate students and volunteers in the fifth floor as well. The fifth floor also houses several meeting areas and collaborative spaces. I expect grad students and post docs to use this space daily for research work, writing, data analysis. We hold lab meetings in room 5107 and you can book the smaller collaborative rooms for working together on projects. The best way to get the most out of this research space will be be here doing research work and interacting with other trainees from the other labs on the same floor. 

\subsection{TA Office}
The 5th floor open office in WIRB should not be used to hold TA office hours, because it is an open office. Graduate students who are working as TAs may also have access to shared space in the Social Science Centre where they can hold office hours. You may also wish to arrange to hold office hours in a set location at Weldon Library. You can book small study rooms for this. 

\subsubsection{Research Space} Our research testing areas are located on the third floor. Room 3138 is the \textbf{Mindfulness Meditation Lab}, which can be used for studies involving meditation, or other similar interventions where a quiet group setting is needed. We also use the general testing rooms on the second and third floor for behavioural testing. These are rooms 2106, 2107, 2112, 2113, 2114, 2115, 2116, 2119, 3106, 3108, 3110 3114, 3158, 3160. These can be booked on the catlab@gmail.com google calendar. The other rooms in the BMI, such as those housing EEG/ERP equipment, psychophysics equipment, eye tracking, and TMS machines, can be used for our research, though in those cases, contact the lead researcher who runs that lab, or ask the BMI equipment manager, Derek Quinlan, on the third floor. 

In addition, we still have some lab space available in the Social Science Centre. The testing rooms in (7242-7246) and houses facilities for data collection that can be booked on the lab calendar. This testing space has room to test up to four participants at a time and can be booked on the lab calendar. This will be helpful in the next few years because it will be easier for undergraduate participants to find this space as the WIRB is still a new facility. Note that we may share this testing area with Ken McRae's lab. 

\section{Lab Communication}

\subsection{Slack Channels}
All lab members should have a Slack account. This will be primary means of communication. Install Slack on your mobile device, work desktop, and laptop, and make sure to customize the notification preferences so that you see what you need to see, but are not overwhelmed. Learn to use the ``@'' feature to get my attention if you need to bring me into a conversation. Make sure you are subscribed to the right channels. The catlabgroup.slack.com account is a paid, premium account so all message and files are archived. 

The lab wiki is run on Slack, on the \#lab\_manual channel. This is where most general information can be found. 

\subsection{Email}
All students have their own email addresses provided by the university, but we have three addresses associated with the lab in general. 
\begin{itemize}
\item catlab@uwo.ca can be used for participant recruitment and communication for the lab in general 
\item mindful@uwo.ca should only be used for study coordination and participant recruitment for any research that is associated with our mindfulness meditation research.
\item catlabuwo@gmail.com is used to access the shared calendars.
\end{itemize}

\subsection{Calendar} The lab calendar (CatLab) is run from our google account, catlabuwo@gmail. The login information is found in the lab wiki (on Slack) and you can either add the entire set of calendars, or just the main one. All lab meetings, trainee meetings, deadlines, and talks should be booked onto this calendar. You can also access room specific 



%%LAB MEETING&&

\section{Lab Meetings} \label{sec:Lab}
We hold a 1.5 hours lab meeting every week. The weekly meeting consists of core lab members (Me, graduate students, postdocsm and honours students when possible). Occasionally, other members will be invited to attend as well. The purpose of these meetings is to discuss current projects, to engage in debate or discussion about topics in the literature, to plan future studies. Mostly, the purpose of the lab meeting is to solve problems. We meet in room 5107 WIRB. I usually schedule lab meetings at a time that is convenient for the maximum number of people. This varies by term, and it may very even by month. We don't usually meet as often in the summer. 

Lab meeting announcements, agendas, and planning will always be in the \#lab\_meeting channel in Slack. 

\subsection {Discussion of Current Projects}
One major purpose of lab meetings is to discuss ongoing projects by team members. This can be anything related to the research program, such as a practice talk for a conference, ongoing data collection, a proposal for a new study, or an idea for a new study. I expect that the the student leading the discussion will prepare an informal presentation of slides, handouts, or notes.

\subsection {Discussion of Current Literature} 
It's also important to stay current with literature. Some of our lab meetings will be set aside for journal club, or for the presentation of interesting and relevant literature. If it is your week to present a paper, we can discuss your choice ahead of time or I can suggest a paper for you. You can choose from papers related to categorization, concepts, learning, cognitive science, methods, computational work, or meditation. There are always good suggestions in the \#goodreads channel in Slack.  

\subsection {Professional Development} From time to time, we will hold workshops on professional development. We can work on your CVs, discuss conferences, job opportunities, etc. 

\subsection{Methods} We will also occasionally use our lab meeting time to run tutorials on statistics software, methodology, programming, writing, how to use OSF, and other related topics. I will usually schedule and run these, but also welcome suggestions from everyone if you particularly skilled in a technique or software, I may ask you to run a tutorial as well.

\section{General Lab Practices}\label{sec:LabPrac}
In this section, I provide an overview of some of the practical aspects of being a student or a volunteer in my lab. This is a long list of general topics. More specific and up to the minute detail can also be found in the \#lab\_manual channel in Slack. This is probably an incomplete list, and I'll add to it.

%Data Management 

\subsection{Open Science} An important aspect of conducting scientific research is the practice of \textit{Open Science} This term might mean different things in different labs, but in my lab there are several best practices that I would like all lab members to follow. 

\begin {itemize}
\item Our lab participates in the Open Science Framework (OSF) and strives to make our work open and accessible to the public. All papers will eventually be available as preprints to ensure public access and will be hosted on PysArXiv, a preprint server hosted by OSF. Public access to our work is not only an example of a good scientific practice, it as also required by most funding agencies (caveat: some of our older papers are hosted in public folders on a server).
\item Graduate students should create an OSF profile so that I can link to it for projects and profiles. OFS is where we host preprints, reprints, data sets, and analysis code. Maintain the public profile accordingly: funders, other scientists, possible employers, and post doc supervisors will see it. 
\item Data will be available on the OSF archive along with the corresponding analysis scripts. 
\item We will strive to meet a minimum of 50\% of publications in open access journals or journals with an open access policy. 
\end{itemize}

\subsection{Record Keeping} Good record-keeping is one of the most important things in science. Although there are a variety of ways to take notes and keep records, it is important to have a centralized and accessible record-keeping system for current projects. Every ongoing research product will have an associated Google Doc that is shared with lab members who are involved in that project.  At the top of the document, is the internal title for the project, the names of the associated lab members, and links to important documents/folders (Manuscript drafts, datafiles, R-code). If you make a change in some code, collect data, edit a document, or anything else related to a project, please record that information in the Google doc with the date and your initials. 

How much detail is required in this record? There's no perfect answer to that question, but the document should be clear enough and detailed  enough so that if we revisited the project several years later, we should be able to find where the manuscript is, where any relevant code is, and where the data is stored. 

\subsection{Document Preparation} After experimenting for a year, I’d like to continue to use Google Docs for writing papers. We use a standard, APA template to do the writing. There is a copy of this template shared with everyone. Assume that the Google Docs version is the core version, but a manuscript version may need to be tweaked for style. Submitted versions will be .docx files or PDF files and saved in the corresponding Google or Dropbox folder. Preprints can be generated as PDFs. Your article preprint can be the same version (APA or similar) that was accepted at the journal, or you can reformat for a nicer appearance. 

\subsection{References and Bibliography} As soon an you start working on a research project, you should learn to use a reference management system. We use Paperpile to manage and store references, because it integrates very well with Google Docs and Chrome. Each full time lab member will have access to the licence. Mendelay is a good (free) option as well that works with Word and can sync with Paperpile.

\subsection{Statistics}
Install the newest versions of R and R Studio and be familiar with how to run and document basic analyses. In particular, learn to use the markdown services (like R Notebooks) to document your code and explain your analyses. We’ll also use JAMOVI for basic analysis. Ensure that you have the newest version installed. JASP is worth considering as well, but I’d rather stick with one. JAMOVI works well with R.

\subsection{Ethics}\label{sec:Ethics}
All of our research involving human participants need to be approved by the \href{http://www.uwo.ca/research/services/ethics/index.html}{Western Research Ethics Board.} We will periodically review all of our open protocols to ensure compliance. These reviews will take place at the start of each academic term (May, Sept, Jan)

\subsection{Google Scholar}
Once you have something indexed or published, create and maintain a Google scholar page. You can link to your co-authors accounts and follow other accounts.

\subsection{Lab Web page}
I run the lab \href{http://mindalab.worpress.com}{website} from WordPress. Please create and maintain your own profile on lab website. Or better yet, just create a personal page that I can linked to. You can use WordPress (free), Google Sites (also free) or any other straightforward website development tool. This is important, because if a prospective postdoc supervisor or prospective employer searched your name, you want that page to come up near the top. It should be clean, professional, and informative.

Don't bother with Research Gate or Academia.edu. They are not open and accessible. In my view, nothing is to be gained by subscribing to their business models.

\subsection{CV}
Maintain an up to date CV on your primary, personal machine but ensure that a current copy is also stored the CV Workshop folder in the lab Google Drive, so that it can be included on the website. 

% \subsection{Weekly updates}
% Good record keeping is essential to doing good science. There are many tools at our disposal, as we each keep out own notebooks, and we have an Evernote notebook that acts as a lab Wiki. At some point, we'll have a lab Wiki. 



\subsection{Software and Hardware}
We use many different software packages and the details for the critical ones should be available on \#lab\_notebook channel in Slack. Below is an incomplete list of important software that full time lab member should have installed or have online access to:
\begin{itemize}
\item MS Office (Word, Excel, Powerpoint)
\item Google Docs, Slides
\item Slack (desktop app and mobile)
\item Paperpile
\item R and RStudio
\item JAMOVI
\item PsychoPy
\item Qualtrics
\end{itemize}
In addition to these above, there are site licences for Matlab, and some shared resources for E-Prime


\section{Conferences}\label{sec:Conf}
One of the best way to present your current research, to find out about other cutting-edge research, and to meet other scientists is at scientific conferences and conventions. There are a number of conferences that the lab is involved with, though we may not all attend every conference. I don't usually attend more than 1-2 a year, because I don't care for long travel, but I encourage students and trainees to present at additional conferences. As a rule, you can find up to date information on conferences in the \#conferences channel on slack.

\subsection{Conference Funding}
A local conference might cost several hundred dollars to attend and a national or international conference might cost over one thousand dollars. I can subsidize at least one conference per year, and the Psychology Department also offer \$250 per student per year. There are also travel scholarships available for many of the conferences. Information about these scholarships is usually available from conference websites. 

To maximize the experience and minimal expense, consider room sharing with other graduate students.

Please save every receipt: hotel, taxi, flight and each meal, etc. These are needed to complete a travel reimbursement. It is your responsibility to complete the reimbursement form. I'll be notified when it's complete and can approve it. Please discuss reimbursement prior to attending, in order to ensure that I have budgeted for this trip. 


\subsection{Common Conferences}
This is a list of conferences that we have attended with some regularity.  There are others, of course, and these will usually be announced in in the \#conferences channels in slack.

\begin{description}
\item [\href{http://www.psychonomic.org}{Psychonomic Society} ] The Psychonomic Society conference is held in November and the abstracts are due in June. As a trainee, you can present a poster and author a talk, but you cannot actually give the talk unless you are a full member of there society (I am). Membership in the Psychonomic society is usually restricted to researchers and faculty. I can author a submission and sponsor a submission, which means that as a lab, we are typically limited to two submissions. I attend the Psychonomics conference almost every year, and it is the one conference that we often try to do as a lab.

\item [\href{http://www.cognitivesciencesociety.org/}{Cognitive Science Society}] The conference is usually in July and the submissions are due in February. Cognitive Science is great for interdisciplinary science and also publishes proceedings for each conference. Trainees are eligible to present papers (spoken presentations) or present posters. I try to attend every few years.

\item  [\href{http://www.psychologicalscience.org/conventions/annual}{Psychological Science (APS)}] This is a large, international conference that covers experimental psychology in general. It is worth attending at least once or twice, as you will get to see some of the most influential psychologists in the world. Presentation opportunities for trainees are usually limited to posters, but some opportunities for presenting are possible if we're involved in a symposium.

\item [\href{https://www.csbbcs.org/}{CSBBCS}] - The Canadian Society for Behaviour, Brain, and Cognitive Science is a great, Canadian focused conference that is held annually at a University in the summer. There are opportunities for trainees to give talks and posters, and there is usually an enjoyable hospitality event. 

\item [\href{https://www.midwesternpsych.org/}{Midwestern Psychological Association}] This is a regional conference held every year in Chicago that is affiliated with the APA. There are opportunities for trainees to give talks and posters, and there is usually an enjoyable hospitality event. 

\end{description}

\section{Authorship}
We are working together to carry out scientific investigations. Part of this process is writing up the work for publication and depending on your role, you may be added as an author on the paper. This is more common for graduate students, but authorship is possible for undergraduate honours students as well. 

\subsection{What counts as a contribution?}
Authors on journal articles and chapters are expected to have made a \textbf{concrete contribution} to the project, and/or paper. For example, the following are all justification for inclusion as an author.
\begin{itemize}
\item You designed one or more experiment in the paper.
\item You wrote the initial draft for a major section.
\item You wrote the entire paper.
\item You designed and carried out the analyses.
\end{itemize}

In many cases, you might help with a projects, but the contribution is not quite enough to warrant inclusion as an author. The following are several examples.
\begin{itemize}
\item You helped to carry out data collection.
\item You created the reference section, table, or figure.
\item You scored a test or created a data set.
\item You helped to proof read or edit.
\end{itemize}
In these cases, you will be acknowledged by name in the paper.

\subsection{Author Order} The order of authorship matters, but there is no consistent agreement in Psychology as to how authors should be ordered. In the Cognitive Science, the convention is that the \textbf{first} and \textbf{last} positions have special meaning, with last author usually being the P.I. or senior investigator on a multi-author projects, and the first author often being the trainee (PhD or Postdoc) who wrote much of the paper. Regardless of position, there is also the \textbf{corresponding author}, which is the person who is the PI and is ultimately responsible for the contents. The corresponding author will  be first or last and, as the name suggests, will correspond with the journal, pay any Open Access fees, correspond with media, and ultimately has to Below of some guidelines that I use.
\begin{itemize}
\item If I designed the experiment and wrote much of the paper, I will serve as first author and corresponding author. 
\item If a trainee carried out some of the research under my supervision, but I designed the study and/or wrote most of the paper, I will serve as first and corresponding author.
\item If a trainee carried out the research under my supervision, and helped to design the study (or designed it entirely) and/or wrote much of the paper, the trainee will serve as first authors and I will serve as final, corresponding author. \textit{Note: this is the most common format in my lab.}
\item For PhD students, if you are publishing your dissertation research and planning to pursue a scientific career, you should act as first and corresponding author. 
\item For PhD students, if you are not interested or able to publish your dissertation research, we can discuss how to proceed. One option is that you will be included as first author, and I will act as final and corresponding author. Another option is that another student may work on the paper, and would take the first author position. This would only happen if you gave approval and agreed that you were not interested in writing up the work for publication. 
\end{itemize}

\section{On Writing}
Writing is one of the most important parts of being a successful scientist and academic. All the well designed experiments, rigorous analysis, and technical achievements will not be worth very much if you cannot write about  them.  Writing is not easy. Writing takes practice. There are several ways to gain this practice.

\begin{description}
\item [Class assignments and course work.] One of the fundamental ways to learn to be a good writer is to engage in written course work, such as thought papers, research proposals, and research papers. Not all courses have a significant writing component, but many do. We also occasionally offer a course on "Scientific Writing".

\item [Lab Reports.] I encourage you to write up short reports for each study that that you run. This is important for several reasons. First, it provides a record of every study we run, regardless of whether we eventually publish it or not. This allows us to track the number of subjects we ran, specific protocol we ran, and preliminary analyses. Second, a short write-up can be used to drop into a manuscripts and expand into a full method section. Finally, it's just good practice for the process of scientific writing. 

\item [Blog Entries.] You should absolutely have a research blog/website. Having a professional web presence is important for many reasons, but one benefit to authoring blog entries is that you gain practice and feedback on describing research for an audience outside the field and outside out department. Consider setting up a Wordpress site which I can link to from the lab website \href{http://mindalab.wordpress.com}{http://mindalab.wordpress.com/}. Publicize your writing. Share your blog entries. It's vitally important to cultivate a thoughtful and engaged presence.

\end{description}





\end{document}